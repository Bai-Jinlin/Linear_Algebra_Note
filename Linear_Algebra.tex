
\documentclass[UTF8]{report}
\usepackage{amsfonts}
\usepackage{amsmath}
\usepackage{amssymb}
\usepackage{ctex}
\author{Jinlin Bai}
\title{Linear Algebra Note}
\begin{document}
	\maketitle
	\tableofcontents
	\newpage
	\newtheorem{define}{定义}
	\newtheorem{theorem}{定理}
	\chapter{线性代数中的线性方程组}
	\begin{itemize}
		\item $a_1 x_1 + a_2 x_2 + \cdots + a_n x_n = b \quad a \textrm{ and } b\in \mathbb{R}$
		\item 相容指线性方程组有一个解或无穷多个解 
		\item 行变换可逆
		\item \begin{define}
			\begin{tabular}{|c|c|}
				\hline
				阶梯型矩阵 & $\begin{bmatrix}
					\blacksquare & * & * & * & * \\
					0 & \blacksquare & * & * & * \\
					0 & 0 & 0 & \blacksquare & *
				\end{bmatrix}$ \\
				\hline
				简化阶梯型矩阵 & $\begin{bmatrix}
					1 & 0 & 0 & * & * \\
					0 & 1 & 0 & * & * \\
					0 & 0 & 1 & * & *
				\end{bmatrix}$\\
				\hline
			\end{tabular}
		\end{define}
		\item \begin{theorem}
			每个矩阵行等于唯一的简化阶梯型矩阵
		\end{theorem}
		\item \begin{define}
			矩阵中的\textbf{主元位置}是$A$中对应与它的阶梯型中的先导元素的位置,\textbf{主元列}是$A$的含有\textit{主元位置}的列
		\end{define}
		\item \begin{define}
			对应主元列的变量称为基本变量,其他变量称为自由变量
			\end{define}
		\item \begin{theorem}[存在与唯一性定理]
			线性方程组相容的必要充分条件是增广矩阵的最右列不是主元列,也就是说,增广矩阵没有形如$\begin{bmatrix} 0 & \cdots & 0 & b \end{bmatrix}\quad b \neq 0$的行
		\end{theorem}
	\end{itemize}
\end{document}